\documentclass[a4paper,12pt]{article}
\usepackage[utf8]{inputenc}
\usepackage[ngerman]{babel}
\usepackage{csquotes}
\usepackage[
	backend=biber,
	style=ieee,
	sorting=none,
]{biblatex}
\usepackage{amsmath}
\usepackage{graphicx}
\usepackage{booktabs}
\usepackage{titlesec}
\usepackage{xcolor}
\usepackage{caption}
\usepackage{wrapfig}
\usepackage{float}
\usepackage{pdflscape}
\usepackage{tabularx}
\usepackage{fancyhdr}
\usepackage{hyperref}
\usepackage{geometry}
\usepackage{bm}

%DEFINES

\def\Autor{Bastian Nagler}
\def\Semester{Wintersemester 2025/26}
\def\Titel{Klatschsensor Ausarbeitung}

\graphicspath{../Bilder/}

%Bibliographie
\addbibresource{literatur.bib}
\ExecuteBibliographyOptions{url=true}

% Seitenränder
\geometry{margin=2.5cm}

%Einrückung der Absätze
\setlength{\parindent}{0pt}
\setlength{\parskip}{6pt}

%leere erste Seite
\fancypagestyle{plain}{
  \fancyhf{} % alles löschen (Header + Footer)
  \renewcommand{\headrulewidth}{0pt}
  \renewcommand{\footrulewidth}{0pt}
}

% Kopfzeile
\pagestyle{fancy}
\setlength{\headheight}{24pt} % Höhe der Kopfzeile anpassen
\fancyhead[L]{\includegraphics[width=4cm]{../Bilder/OTH_Logo_1lineB_pos_b.png}} % Bild auf der linken Seite
\fancyhead[R]{\Titel} % Titel auf der rechten Seite
 
% Fußzeile
\fancyfoot[R]{\thepage} % Seitenzahl in der Fußzeile
\fancyfoot[C]{} % Keine Seitenzahl in der mittleren Fußzeile
\fancyfoot[L]{\Autor}
\renewcommand{\footrulewidth}{0.4pt}


\begin{document}
	\begin{titlepage}
		\thispagestyle{empty}

			\begin{center}
				\vspace*{\stretch{0.5}}
				\includegraphics[width=0.5\textwidth]{../Bilder/OTHR_FakEI_Logo.png}\\
				\vspace*{\stretch{0.25}}
				\Huge
				\textsc{Ausarbeitung}\\
				\textsc{Klatschsensor \hspace*{1pt} (V 1)}\\
				\large
				\vspace*{\stretch{0.5}}
				\vspace*{\stretch{0.5}}

		\vspace*{\stretch{2}}
		{\renewcommand{\arraystretch}{1.5}
		\begin{tabular}{l l}
			\href{https://github.com/BastianNagler/PMT1_Klatschsensor}{Erstellt von:}  & \hspace{4cm} Bastian Nagler\\
			%Matrikelnummer:  & \hspace{4cm}\MatNr\\
			Letzter Stand:  & \hspace{4cm} \textsc{\Semester}  \\
			Lizenz:  & \hspace{4cm} GPLv3
		\end{tabular}
		}
		\vspace*{\stretch{1}}

		\end{center}
	\end{titlepage}

	\newpage

	\tableofcontents
	\newpage

	%INPUTS

	\section{Versuchseinführung}

Im Versuch Klatschsensor wird eine bereitgestellte Schaltung analysiert, die ein akustisches Signal --- etwa ein Klatschen --- erfasst,
 verstärkt und damit eine LED schaltet.
Dabei werden grundlegende messtechnische und elektronische Konzepte praktisch angewendet.

Zentrale Bestandteile sind ein Elektretmikrofon, ein Operationsverstärker zur Signalverstärkung,
 eine Diode zur Einweggleichrichtung sowie ein Komparator, der die Spannung mit einer Referenz vergleicht und bei Überschreitung eine LED ansteuert.

Durch die Arbeit mit dem Digitaloszilloskop und dem Funktionsgenerator wird der Umgang mit typischen Messgeräten vertieft.
Zudem wird die Funktion und Anwendung des invertierenden Verstärkers und des Komparators untersucht.
Ein weiterer Schwerpunkt liegt in der Analyse von Lade- und Entladevorgängen eines Kondensators, die das zeitliche Verhalten des Sensors bestimmen.
Des weiteren wird die Klassifizierung von Widerständen anhand ihres Farbcode-Systems über den gesamten Versuch hinweg behandelt.\ \cite{Beschreibung}

Außerdem wurde während der Versuchsdurchführung großer Wert auf die kritische Betrachtung der Messergebnisse gelegt und es wurden unter den Gruppen Fehlerquellen identifiziert und diskutiert.

	\section{Bantwortung der Vorbereitungsfragen}

\textbf{Frage 1: Welchen Wert hat die Verstärkung des invertierenden Verstärkers in der Grundkonfiguration?}

\begin{align*}
    V = - \frac{R_3}{R_2} = - \frac{56~k\Omega}{2,2~k\Omega} \approx -25,45
\end{align*}

\textbf{Frage 2: Wie groß ist der garantierte Ausgangsspannungsbereich des realen OPV TL081C bei einer Versorgungsspannung von \(\pm 15~V\) und einer Belastung mit \(R_L = 2~k\Omega\)}

Aus dem Datenblatt: \(\pm 10~V\)

\textbf{Frage 3: Wie groß wäre der Ausgangsspannungsbereich bei einem idealen OPV unter sonst gleichen Bedingungen?}

Der Ausgangsspannungsbereich entspräche genau der Versorgungsspannung, also \(\pm 15~V\)

\textbf{Frage 4: Welche Spannung \(\bm{U_{ref}}\) liegt am Komparatoreingang ``-'' an?}

\begin{align*}
    U_{ref} = \frac{R_7}{R_7 + R_{11}} \cdot U = \frac{10~k\Omega}{10~k\Omega + 1~M\Omega} \cdot 10~V \approx 99~mV
\end{align*}

\textbf{Frage 5: Welche Entladezeitkonstante des Kondensators ist mit der Grundkonfiguration eingestellt?}

\begin{align*}
    \tau = R_5 \cdot C_2 = 180~k\Omega \cdot 0,47~\mu F = 84,6~ms
\end{align*}

\textbf{Frage 6: Wie lange würde es theoretisch dauern, bis die LED nach einem Klatschimpuls wieder ausgeht, wenn \(U_{ref} = 0~V\) wäre?}

Es würde rein theoretisch unendlich lange dauern, da die Spannung des Kondensators beim Entladen nie 0~V erreicht und somit nie unter die Referenzspannung fällt, 
 wodurch der Komparator immer auf die positive Versorgungsspannung am Ausgang schaltet.

\textbf{Frage 7: Wie setzt sich der Farbcode für einen \(360~k\Omega\) Widerstand zusammen?}

Annahme: 5\% Toleranz

4-Ring-Farbcode: orange-blau-gelb-gold\\
5-Ring-Farbcode: orange-blau-schwarz-orange-gold
	
	\section{Versuchsaufbau}\label{section:aufbau}

\subsection{Schematischer Aufbau}\label{subsection:schematischer_aufbau}

\begin{figure}[H]
    \includegraphics[width=\textwidth]{../Bilder/Aufbau.png}
    \caption{Schematischer Aufbau der Schaltung~\cite{Beschreibung}}\label{fig:aufbau}
\end{figure}

Der in der Abbildung~\ref{fig:aufbau} rot dargestellte Bereich zeigt das Mikrofon inklusive einer Gleichspannungsquelle, welche eine Arbeitspunktverschiebung des Mikrofons bewirkt.
Diese Gleichspannung dringt allerdings nicht in den weiteren Signalverarbeitungsweg ein, da sie durch den Kondensator C1 gefiltert wird.

Da die Signale des Mikrofons sehr schwach sind, müssen diese zunächst verstärkt werden.
Hierfür wird ein invertierender Operationsverstärker (gelber Bereich) verwendet.
Die Verstärkung des OPVs wird durch die Widerstände \( R_2 \) und \( R_3 \) bestimmt und beträgt in der Grundkonfiguration:~\cite{Beschreibung}
\begin{align*}
    V = - \frac{R_3}{R_2} = - \frac{56~k\Omega}{2,2~k\Omega} \approx -25,45
\end{align*}

Das verstärkte Signal wird anschließend durch den Einweggleichrichter, implementiert durch die Diode \(D_1\), gleichgerichtet.

Die in grün gekennzeichneten Bauteile (\(C_2\) und \(R_5\)) bewirken eine Glättung des gleichgerichteten Signals,
 um die Spannungslücken des Signals zu überbrücken und außerdem auch um die Haltezeit der LED am Ende zu erhöhen.
Dies geschieht, indem der Kondensator durch das Signal aufgeladen wird und sich daraufhin langsam über \(R_5\) entlädt.

Diese Ausgangsspannung wird dann über \(R_6\) an den positiven Eingang eines Komparators, welcher durch einen OPV realisiert wird, weitergeleitet.
Der Komparator vergleicht die Eingangsspannung mit einer Referenzspannung, die über das Spannungsteilernetzwerk aus \(R_7\) und \(R_{11}\) festgelegt wird.
Ist die Eingangsspannung \(U_{EK}\) größer als die Referenzspannung \(U_{Ref}\), schaltet der Komparator seinen Ausgang auf eine Spannung nahe der Versorgungsspannung.
Ist \(U_{EK} < U_{Ref}\), so beträgt die Ausgangsspannung einen Wert nahe der negativen Versorgungsspannung.

Der Ausgang des Komparators schaltet über den Vorwiderstand \(R_9\) die LED \(D_2\),
 welche aufgrund der Eigenschaft der Diode nur bei einer positiven Spannung des Komparator-Ausgangs aufleuchtet.

\subsection{Realer Aufbau}\label{subsection:realer_aufbau}

\begin{figure}[H]
    \includegraphics[width=\textwidth]{../Bilder/realer_aufbau.jpg}
    \caption{Realer Aufbau der Schaltung}\label{fig:realer_aufbau}
\end{figure}

Der reale Aufbau wirkt unspektakulär. Die gesamte Schaltung ist nicht einsehbar unter einer Platte realisiert, lediglich die Widerstände \(R_3\) und \(R_5\) liegen oberhalb, um diese verändern zu können.
Außerdem sind Anschlüsse für die Versorgungsspannung (-10~V, 0~V und +10~V) sowie Anschlüsse für das Oszilloskop an den gewünschten Messpunkten vorhanden.

In Abbildung~\ref{fig:realer_aufbau} sind im wesentlichen vier verwendete Geräte zu erkennen:

\begin{enumerate}
    \item Netzteil mit zwei seperaten Spannungsquellen
    \item Eigentliche Schaltung des Klatschsensors
    \item Funktionsgenerator Rigol DG1012
    \item Oszilloskop Rigol DS1102E
\end{enumerate}

Die Spannungsquellen werden innerhalb des Netzteils in Serie geschaltet, um die benötigte bipolare Spannung von \(\pm 10~V\) zu erhalten.
Dabei wird der Minuspol auf -10~V, der gemeinsame Pol auf 0~V und der Pluspol der Quellen auf +10~V gelegt.

	\section{Veruchsdurchführung und Auswertung}\label{section:durchführung}

\subsection{Vorbereitung}\label{subsection:vorbereitung}

Zunächst wird \(R_3\) mit einem \(56~k\Omega\) und \(R_5\) mit einem \(180~k\Omega\) Widerstand bestückt.
Außerdem wird die bipolare Spannungsquelle, wie in Abschnitt~\ref{subsection:realer_aufbau} beschrieben, aufgebaut.

\subsection{Klatschsignal}\label{subsection:klatschsignal}

Die erste Messung besteht darin, ein Klatschsignal auf dem Oszilloskop darzustellen.
Dazu wird der Schalter \(S_1\) auf die obere Position gestellt und das Oszilloskop mit dem Messpunkt ``Audiosignal'' verbunden.
Wichtig hierbei ist, den Trigger des Oszilloskops auf den ``single'' Modus zu stellen, wodurch nach einmaligen Auslösen des Triggers die Messung gestoppt wird und das Messergebnis erhalten bleibt.
Wäre der Trigger im normalen Modus, so würde der Speicher des Oszilloskops nach einer kurzen Zeit überlaufen und das Signal wäre nicht mehr sichtbar.
Anschließend wird nahe dem Mikrofon geklatscht, wodurch ein Signal am Oszilloskop dargestellt wird.\ (vgl.\ Abb.~\ref{fig:0_f1})

\begin{figure}[H]
    \centering
    \includegraphics[width=0.7\textwidth]{../Bilder/0_F1_ohneCursor.jpg}
    \caption{Klatschsignal am Oszilloskop dargestellt}\label{fig:0_f1}
\end{figure}

Aus diesem Signal werden daraufhin je zwei möglichst verschiedene Werte für die Amplitude und Frequenz mithilfe des Cursors (vgl.\ Abb.~\ref{fig:1_f2}) ermittelt.\ (vgl.\ Tab.~\ref{tab:klatschsignal})

\begin{figure}[H]
    \centering
    \includegraphics[width=0.7\textwidth]{../Bilder/1_F2.jpg}
    \caption{Frequenzermittlung mit dem Cursor}\label{fig:1_f2}
\end{figure}

\begin{table}[H]
    \centering
    \begin{tabular}{ccc}
        \toprule
        Messung & Amplitude in V & Frequenz in Hz \\
        \midrule
        1 & 1,17 & 5000 \\
        2 & 0,16 & 1670 \\
        \midrule
        Mittelwert & 0,665 & 3335 \\
        \bottomrule
    \end{tabular}
    \caption{Gemessene Amplituden und Frequenzen des Klatschsignals}\label{tab:klatschsignal}
\end{table}

Das Hauptziel dieser Messung war es, zu bestätigen, ob die vorgegebenen Werte für die Funktion des Funktionsgenerators in der richtigen Größenordnung liegen.

Die Messung einer Schaltung funktioniert nur sinnvoll mit einem determinierten Signal, um reproduzierbare Ergebnisse zu erhalten.
Hierbei muss allerdings auf die Sinnhaftigkeit dieses Signals geachtet werden, d.h.\ das Signal sollte in etwa einem realen Klatschen entsprechen.

Während zwar mit \(665~mV\) eine sehr viel größerer Wert als die vorgegebenen \(300~mV\) erreicht wurden, wird trotzdem eine Funktion mit Amplitude \(300~mV\) gewählt,
 da die Messung sehr nahe am Mikrofon stattfand und außerdem eine Vergleichbarkeit der Ausarbeitungen gefordert ist.
Die gemessene Frequenz von \(3335~Hz\) liegt nahe an den vorgegebenen \(3000~Hz\) und bestätigt somit die Sinnhaftigkeit des vorgegebenen Wertes.

\subsection{Invertierender Verstärker}\label{subsection:inverting_amp}

Wie in Abschnitt~\ref{subsection:klatschsignal} beschrieben, wird nun ein determiniertes Signal verwendet mit Amplitude \(300~mV\) und Frequenz \(3~kHz\).
Außerdem wird am Funktionsgenerator ein Burst-Modus konfiguriert, mit einem Cycle von 1 und einer Periode von 1 Sekunde.

Zunächst wird das Signal direkt in das Oszilloskop eingespeist, um die Korrektheit zu überprüfen.
Daraufhin wird der Schalter \(S_1\) auf die untere Position --- die BNC-Buchse --- gestellt und diese mit dem Funktoionsgenerator verbunden.
Außerdam kann am Oszilloskop jetzt wieder ein normaler Trigger-Modus verwendet werden,
 da das neue Signal periodisch ist und die gesamte Periode von 1 Sekunde problemlos im Oszilloskop zwischengespeichert werden kann.

Ziel der Messreihe ist es, die Verstärkung des invertierenden Verstärkers für verschiedene \(R_3\) zu bestimmen.
Dazu wird Kanal 1 des Oszilloskops mit dem Messpunkt ``Audio\-signal'', Kanal 2 mit ``Verstärkerausgang'' verbunden.
Um die peak-to-peak Spannungen zu messen, wird die integrierte Measure-Funktion verwendet.

Die gemessene Verstärkung \(V\) lässt sich über folgende Formel berechnen:

\begin{align*}
    V = \frac{u_{amplified}}{u_{Audio}} = - \frac{R_3}{R_2}
\end{align*}

Hierbei ist allerdings zu beachten, dass \(u_{amplified}\) im Vergleich zu \(u_{Audio}\) ein invertiertes Signal ist (vgl.\ Abb.~\ref{fig:inverting_amp}).
In der Formel wird dies berücksichtigt, indem \(u_{amplified}\) ein negatives Vorzeichen erhält.

\begin{figure}[H]
    \centering
    \includegraphics[width=0.7\textwidth]{../Bilder/Inverting_Amp.jpg}
    \caption{Invertierender Verstärker am Oszilloskop}\label{fig:inverting_amp}
\end{figure}

Außerdem wird die Verstärkung \(V_{theoretisch}\) ebenfalls über die Widerstände \(R_2\) und \(R_3\) berechnet und mit den gemessenen Werten verglichen.

\begin{table}[H]
    \centering
    \begin{tabular}{ccccc}
        \toprule
        \(R_3\) in \(k\Omega\) & \(u_{Audio}\) in \(V\) & \(u_{amplified}\) in \(V\) & \(V_{gemessen}\) & \(V_{theoretisch}\) \\
        \midrule
        2,2 & 0,294 & -0,294 & -1,00 & -1,00\\
        3,6 & 0,295 & -0,476 & -1,61 & -1,64\\
        6,8 & 0,294 & -0,912 & -3,10 & -3,09\\
        11 & 0,297 & -1,46 & -4,92 & -5,00\\
        51 & 0,294 & -6,84 & -23,27 & -23,18\\
        56 & 0,294 & -7,48 & -25,44 & -25,45\\
        82 & 0,298 & -11,0 & -36,91 & -37,27\\
        100 & 0,294 & -12,8 & -43,54 & -45,45\\
        110 & 0,296 & -13,4 & -45,27 & -50,00\\
        \bottomrule
    \end{tabular}
    \caption{Messung der Verstärkung des invertierenden Verstärkers für verschiedene \(R_3\)}\label{tab:inverting_amp}
\end{table}

\begin{figure}[H]
    \centering
    \includegraphics[width=0.8\textwidth]{../Bilder/Inverting_Amp_Plot.jpg}
    \caption{Grafische Darstellung der Verstärkung des invertierenden Verstärkers}\label{fig:inverting_amp_plot}
\end{figure}

Stellt man diese Werte betragsmäßig grafisch dar (vgl.\ Abb.~\ref{fig:inverting_amp_plot}), so fällt vor allem bei linearer Darstellung auf,
 dass bei hohen Widerstandswerten von \(R_3\) die Abweichung zwischen gemessener und theoretischer Verstärkung immer größer wird.

Dies kann durch die niederohmige Last des OPVs erklärt werden.
Laut Datenblatt ist das Verhalten des OPV bei Lastwiderständen unter \(2~k\Omega\) nicht spezifiziert~\cite{OPV}.
Da der Kondensator \(C_2\) zu Beginn des Aufladevorgangs eine geringe Impedanz besitzt, kann im ersten Moment die gesamte Last als \(R_4 = 330~\Omega\) angenommen werden.
Somit können die Ausgangsspannungen bei hoher Verstärkung und niederohmiger Last nicht mehr garantiert werden, 
 da der OPV zum Eigenschutz die Verstärkung reduziert, um zu hohe Ströme durch ihn zu vermeiden.
Es müsste ein Strom von \(\frac{10~V}{330~\Omega} = 30~mA\) aufgebracht werden, der OPV kann allerdings mit der verwendeten Spannungsquelle maximal
 \(\frac{10~V}{2000~\Omega} = 5~mA\) liefern.

\begin{figure}[H]
    \centering
    \includegraphics[width=0.7\textwidth]{../Bilder/Inverting_Amp_110k.jpg}
    \caption{Invertierender Verstärker mit \(R_3 = 110~k\Omega\)}\label{fig:inverting_amp_110k}
\end{figure}

Dies wird unter anderem auch in der Spannungskurve von \(u_{amplified}\) des Oszilloskops bei \(R_3 = 110~k\Omega\) deutlich (vgl.\ Abb.~\ref{fig:inverting_amp_110k}).
Diese flacht ab einem kritischen Wert ab und steigt nur weiter, da sich der Kondensator \(C_2\) auflädt und durch dessen steigende Impedanz die Last erhöht wird und somit auch der 
 Strom verringert wird.

Auffällig hierbei ist allerdings auch, dass nur die positive Halbwelle des Signals abgeschnitten wird. 
Dies liegt daran, dass aufgrund der nachfolgenden Diode, welche bei negativer Halbwelle sperrt, ein vernachlässigbarer Strom im Picoampere"~Bereich fließt, 
 hier greift also kein Schutzmechanismus des OPVs.

\subsection{Einweggleichrichtung}\label{subsection:einweggleichrichtung}

Nun sei wieder \(R_3 = 56~k\Omega\).
Außerdem wird Kanal 1 des Oszilloskops mit dem Messpunkt ``Verstärkerausgang'' und Kanal 2 mit ``Gleichgerichtetes'' verbunden.

\begin{figure}[H]
    \centering
    \includegraphics[width=0.7\textwidth]{../Bilder/Einweggleichrichtung.jpg}
    \caption{Einweggleichrichtung am Oszilloskop}\label{fig:einweggleichrichtung}
\end{figure}

Während einerseits das Erwartete passiert, dass die negative Halbwelle des Signals abgeschnitten wird (vgl.\ Abb.~\ref{fig:einweggleichrichtung}),
 fällt andererseits auf, dass die positive Halbwelle eine erkennbare Absenkung erfährt.

Mithilfe der exportierten CSV-Daten des Oszilloskops kann die Spannungsdifferenz zwischen dem verstärkten Signal und dem gleichgerichteten Signal auf etwa \(0,76~V\) bestimmt werden.
Dieser Spannungsabfall ist auf die Flussspannung der Diode zurückzuführen, welche einen typischen Wert von etwa \(0,7~V\) hat.~\cite{Diode}
Die hier gemessene Spannung von \(0,76~V\) liegt in einem sinnvollen Bereich, um als Flussspannung der Diode interpretiert zu werden.

Des Weiteren fällt auf, dass die positive Halbwelle nicht auf 0~V zurückfällt, sondern stattdessen einen Restwert von etwa \(1,2~V\) behält.
Dies kann durch den Kondensator \(C_2\) erklärt werden. 
Dieser lädt sich während der positiven Halbwelle auf und entlädt sich nur langsam wieder über den Widerstand \(R_5\).
Das Potential hinter der Diode wird somit auf einem höheren Niveau gehalten, während das Potential vor der Diode auf \(0~V\) absinkt, da die Diode keine Spannung in Sperrrichtung durchlässt.
Somit entsteht der beobachtete Offset von etwa \(1,2~V\).

Stellt man am Oszilloskop einen längeren Zeitbereich ein, so erkennt man nach der positiven Halbwelle ein langsames Abfallen der Spannung (vgl.\ Abb.~\ref{fig:einweggleichrichtung_zoom}).
Der Verlauf des Abfalls erinnert stark an eine Entladekurve eines Kondensators, was die vorherige Erklärung weiter untermauert.

\begin{figure}[H]
    \centering
    \includegraphics[width=0.7\textwidth]{../Bilder/Einweggleichrichtung_Zoom.jpg}
    \caption{längerer Zeitbereich der Einweggleichrichtung am Oszilloskop}\label{fig:einweggleichrichtung_zoom}
\end{figure}

\subsection{Diodenspannung}\label{subsection:diodenspannung}

Als nächstes wird die Spannung über die Diode gemessen. 
Dazu wird der Pluspol eines Messkanals mit dem Messpunkt ``Verstärkerausgang'' und der Minuspol mit dem Messpunkt ``Gleichgerichtetes'' verbunden.
Dadurch erhält man folgende Darstellung am Oszilloskop (vgl.\ Abb.~\ref{fig:diodenspannung}).

\begin{figure}[H]
    \centering
    \includegraphics[width=0.7\textwidth]{../Bilder/Diodenspannung.jpg}
    \caption{Spannung über der Diode am Oszilloskop}\label{fig:diodenspannung}
\end{figure}

Auf dem Bild erkennt man deutlich, dass während der negativen Halbwelle des Verstärker\-ausgangs die gesamte Spannung von etwa 3,72~V  an der Diode abfällt.
Außerdem fällt auf, dass die Flussspannung der Diode während der positiven Halbwelle jetzt etwa 0,84~V beträgt. 

Diese Messung ist allerdings mit äußerster Vorsicht zu genießen, da das Oszilloskop in dieser Konfiguration einen Schluss nach Masse darstellt.
Die Masse des Oszilloskops, des Funktionsgenerators und des Netzteils sind über den Schutzleiter miteinander verbunden und geerdet.
Dadurch wird an den Messpunkt ``Gleichgerichtetes'' durch das Oszilloskop eine Verbindung nach Masse hergestellt und die Schaltung wesentlich verändert.
Es liegt somit die gesamte Spannung des Verstärkerausgangs nur an der Diode an. 
Ohne einen Schutzmechanismus des OPVs könnte dies zu einer Beschädigung der Schaltung führen, da annähernd ein Kurzschluss vorliegt.

Ersichtlich wird dieser Messfehler auch daran, dass die LED der Schaltung in dieser Konfiguration nicht leuchtet.

Um die Messung korrekt durchzuführen, gibt es mit den verfügbaren Mitteln im Wesentlichen zwei Möglichkeiten:
\begin{itemize}
    \item Messung der Spannungen an den einzelnen Punkten relativ zur Masse und anschließende Berechnung der Differenz.
    \item Verwendung eines isolierten Oszilloskops, welches keine Verbindung zur Masse herstellt. Dies wäre z.B.\ über einen Trenntrafo oder ein batteriebetriebenes Oszilloskop möglich.
\end{itemize}

Hier wird die erstere Methode gewählt.\ (vgl.\ Abb.~\ref{fig:diodenspannung_math})

\begin{figure}[H]
    \centering
    \includegraphics[width=\textwidth]{../Bilder/Diodenspannung_math.jpg}
    \caption{Spannung über der Diode am Oszilloskop}\label{fig:diodenspannung_math}
\end{figure}

Legt man hier eine konstante Funktion durch die Maxima der Diodenspannung, so ergibt sich diese als ca.\ \(f(x) = 0,76~V\),
 was der Flussspannung aus Abschnitt~\ref{subsection:einweggleichrichtung} entspricht.
Dieser Wert hat sich somit durch zwei unabhängige und korrekte Messungen bestätigt.

\subsection{Lade-/Entladekennlinie}

Nun wird der Funktionsgenerator auf einen peak-to-peak Wert von 3~V und eine Burst-Periode von 3 Sekunden eingestellt um eine vollständige Entladung des Kondensators zu ermöglichen.
Außerdem wird das Oszilloskop wieder an die gemeinsame Masse und den Messpunkt ``Komparatoreingang'' angeschlossen.

Dadurch lässt sich nun bei korrekter Skalierung die Entladekurve eines Kondensators betrachten (vgl.\ Abb.~\ref{fig:kondensator_entladen}).
Konkret handelt es sich hier um den Kondensator \(C_2\), welcher sich über den Widerstand \(R_5\) gegen Masse entlädt.

\begin{figure}[H]
    \centering
    \includegraphics[width=0.7\textwidth]{../Bilder/Kondensator_Entladen.jpg}
    \caption{Entladekurve des Kondensators mit \(R_5 = 180~k\Omega\)}\label{fig:kondensator_entladen}
\end{figure}

Aus diesen Messwerten lässt sich \(\tau\) am Oszilloskop dadurch bestimmen, indem die Stelle gesucht wird,
 bei der die noch am Kondensator anliegende Spannung der Spannung nach \(t = 1 \cdot \tau\) entspricht.
Dieser Wert lässt sich wiefolgt berechnen~\cite{Hipp}:

\begin{align*}
    U_C(t) &= U_0 \cdot e^{-\frac{t}{\tau}} \\
    U_C(\tau) &= U_0 \cdot e^{-\frac{\tau}{\tau}} = \frac{U_0}{e} \approx 0,37 \cdot U_0 \\
    \text{mit } U_0 = 3,3~V: \qquad U_C(\tau) &= 3,3~V \cdot 0,37 \approx 1,22~V
\end{align*}

Der Wert für \(U_0\) wird hierbei mittels des Cursors am Maximum der Spannung abgelesen.

Diese Messung wird nun für unterschiedliche Werte von \(R_5\) durchgeführt und graphisch dargestellt:

\begin{table}[H]
    \centering
    \begin{tabular}{cc}
        \toprule
        \(R_5\text{ in }k\Omega\) & \(\tau\text{ in ms}\) \\
        \midrule
        20 & 9,44 \\
        68 & 30,8 \\
        180 & 76 \\
        390 & 135 \\
        820 & 214 \\
        1000 & 238 \\
        \(\infty\)  & 476 \\
        \bottomrule
    \end{tabular}
    \caption{Messwerte für \(\tau\) bei variierendem \(R_5\)}\label{tab:tau}
\end{table}

Der Wert \(R_5 = \infty\) wird angenähert, indem der Widerstand \(R_5\) gänzlich aus der Schaltung weggelassen wird und die beiden Klemmen im Leerlauf zueinander bleiben.

\begin{figure}[H]
    \centering
    \includegraphics[width=\textwidth]{../Bilder/tau_plot_normal.jpg}
    \caption{Funktion für \(\tau\) in Abhängigkeit von \(R_5\)}\label{fig:tau_normal}
\end{figure}

Der Messpunkt für \(R_5 = \infty\) wird aus dem Graphen natürlich weggelassen, da eine sinnvolle Darstellung nicht möglich ist. 

Der Zusammenhang zwischen \(\tau \text{, } R_5 \text{ und } C_2\) ergibt sich durch folgende Formel:

\begin{align*}
    \tau = R_5 \cdot C_2
\end{align*}

Diese Formel stellt einen linearen Zusammenhang dar, welcher in Abbildung \ref{fig:tau_normal} nicht zu erkennen ist. 
Außerdem sollte laut der Formel für \(R_5 \rightarrow \infty\) gelten: \(\tau \rightarrow \infty\)
Auch dieser Zusammenhang ist in der Tabelle \ref{tab:tau} nicht zu erkennen.

Dadurch lässt sich aufgrund sehr großer Fehler ein systematischer Fehler in der ersten Messung vermuten.
Berechnet man aus \(\tau(R_5 \rightarrow \infty)\) den trotzdem noch vorhandenen Widerstand, so ergibt sich dieser als:

\begin{align*}
    R_{Fehler} = \frac{\tau(R_5 \rightarrow \infty)}{C_2} = \frac{476~ms}{0,47~\mu F} \approx 1~M\Omega
\end{align*}

Dieser ``vergessene'' Widerstand lässt sich durch den Widerstand des Oszilloskops begründen. 
Dieses hat nämlich, wie neben den BNC-Anschlüssen zu erkennen ist, einen Widerstand von \(1~M\Omega\) zwischen Plus- und Minuspol.
Dieser Widerstand liegt parallel zu \(R_5\), da er die selben zwei Potentiale wie \(R_5\) miteinander verbindet.
Berechnet man nun den neuen Gesamtwiderstand \(R_C\) parallel zum Kondensator und damit auch das gemessene \(\tau_{berechnet}\), so ergibt sich folgende Messreihe:

\begin{table}[H]
    \centering
    \begin{tabular}{cccc}
        \toprule
        \(R_5\text{ in }k\Omega\) & \(R_C\text{ in }k\Omega\) & \(\tau_{berechnet}\text{ in ms}\) & \(\tau_{gemessen}\text{ in ms}\)\\
        \midrule
        20 & 19,6 & 9,2 & 9,44 \\
        68 & 63,7 & 29,9 & 30,8 \\
        180 & 152,5 & 71,7 & 76 \\
        390 & 280,6 & 132 & 135 \\
        820 & 450,6 & 212 & 214 \\
        1000 & 500 & 235 & 238 \\
        \(\infty\) & 1000 & 470 & 476 \\
        \bottomrule
    \end{tabular}
    \caption{Messwerte für \(\tau\) bei variierendem \(R_5\)}\label{tab:tau_correct}
\end{table}

Die Abweichungen der Messungen sind jetzt nur noch minimal und z.T.\ auf die Rundung der Größen in den Zwischenschritten zurückzuführen.
Dadurch kann man mit großer Sicherheit sagen, dass das Oszilloskop die Hauptfehlerquelle der falschen Messungen war.

Stellt man diese neuen Werte nun graphisch dar, ergibt sich folgende Abbildung:

\begin{figure}[H]
    \centering
    \includegraphics[width=\textwidth]{../Bilder/tau_plot_corrected.jpg}
    \caption{Funktion für \(\tau\) in Abhängigkeit von \(R_C\)}\label{fig:tau_corrected}
\end{figure}

Als eine Nebenbetrachtung soll außerdem die Ladekurve des Kondensators betrachtet werden.
Diese ergibt sich wiefolgt:

\begin{figure}[H]
    \centering
    \includegraphics[width=0.7\textwidth]{../Bilder/Ladekurve.jpg}
    \caption{Ladekurve von \(C_2\)}\label{fig:ladekurve}
\end{figure}

Auffällig ist hierbei, dass sich keine typische exponentielle Funktion~\cite{Hipp} als Ladekurve ergibt, sondern annähernd eine Gerade.

Dies lässt sich durch die Eingangsfunktion erklären, welche hier eine einzelne betrags\-mäßige Sinus-Kurve ist, wobei Spannungen unter 0,7~V auf 0~V abfallen, aufgrund der Diode.
Der typische Verlauf der Ladekurve ergibt sich allerdings nur, bei einer Sprungfunktion der Eingangsspannung.
Deshalb kann diese Ausgangsfunktion nicht bei einer solchen Eingangsfunktion erwartet werden. 

Wendet man nun allerdings die Grundlagen der SuS-Vorlesung auf diese Signal"=Konstellation an, so lässt sich der gemessene Verlauf theoretisch begründen.
Wie bereits erwähnt, die Funktion \(U(t) = U_0 \cdot (1 - e^{-\frac{t}{\tau}})\) ergibt sich als die Sprungantwort \(g(t)\) eines RC-Systems.
Die Impulsantwort \(h(t)\) des Systems ergibt sich als die zeitliche Ableitung der Sprungantwort:
\begin{align*}
    h(t) = \dot{g(t)} = \frac{dg(t)}{dt} = \frac{U_0}{\tau} \cdot e^{-\frac{t}{\tau}}
\end{align*}
Faltet man diese Impulsantwort nun mit dem Eingangssignal \(x(t)\) unserer Messung, so ergibt sich folgendes Ausgangssignal \(y(t)\):~\cite{SUS}

\begin{figure}[H]
    \centering
    \includegraphics[width=\textwidth]{../Bilder/matlab.jpg}
    \caption{Herleitung der Ladekurve des Kondensators \(C_2\)}\label{fig:matlab}
\end{figure}

Hier lässt sich eine sehr viel größere Ähnlichkeit zwischen der theoretischen und der gemessenen Aufladekurve feststellen.
Die gemessene Kurve wirkt allerdings eher als eine Gerade.
Der Unterschied der beiden Kurven kann durch die unterschiedliche zeitliche Skalierung der Darstellung und auch der Auflösung erklärt werden.
Die Darstellung des Oszilloskops hat eine Auflösung von:
\begin{align*}
    \frac{100~\frac{\mu s}{div}}{25~\frac{Pixel}{div}} \Rightarrow 4~\mu s
\end{align*}

Während der Plot der theoretischen Kurve eine bessere Auflösung von \(2~\mu s\) aufweist, wodurch die kleinen ``Rundungen'' der Kurve im Plot mehr herausstechen.

Die Skalierung der Kurven hat einen ähnlichen Effekt. Abweichungen von einer Geraden gehen dadurch in der Auflösung unter.

\subsection{Komparatorausgang}\label{subsection:komparatorausgang}

Nun wird zusätzlich Kanal 2  des Oszilloskops mit dem Messpunkt ``Komparatorausgang'' verbunden.

Dadurch ergibt sich folgende Messung:

\begin{figure}[H]
    \centering
    \includegraphics[width=0.7\textwidth]{../Bilder/Komparator.jpg}
    \caption{Darstellung des Komparatorausgangs und der Entladekurve von \(C_2\)}\label{fig:komparator}
\end{figure}

Zunächst lässt sich die normale Verhaltensweise des Komparators beobachten.
Ist die Eingangsspannung kleiner als die Referenzspannung, so liegt am Ausgang eine negative Spannung an.
Steigt die Eingangsspannung nun über die Referenzspannung, so liegt am Ausgang eine positive Spannung an, was zum Leuchten der LED führt.

Fällt die Eingangsspannung nun allerdings unter die Referenzspannung, so bleibt der Ausgang noch für eine gewisse Zeit auf die positive Spannung geschalten.
Dies ist ein Verhalten, welches nicht durch einen idealen Komparator beschrieben wird.
Stattdessen fällt der Ausgang erst ab, wenn die Eingangsspannung ca. 80~mV erreicht hat, nicht unwesentlich niedriger als die Referenzspannung \(U_ {Ref} = \frac{R_7}{R_7+R_{11}} \cdot U =  99~mV\).

Grund hierfür könnte sein, dass die Referenzspannung aufgrund der Toleranzen der Widerstände \(R_7\) \(R_{11}\) nicht genau 99~mV entspricht.
Hierfür wird eine Fehlerrechnung durchgeführt:
\begin{align*}
    \Delta U_{Ref}  &= \frac{\partial U_{Ref}}{\partial R_7} \cdot \Delta R_7 + \frac{\partial U_{Ref}}{\partial R_{11}} \cdot \Delta R_{11} \\
                    &= \frac{R_{11}}{{\left(R_7 + R_{11}\right)}^2} \cdot U_{in} \cdot \Delta R_7 + \frac{R_7}{{\left(R_7 + R_{11}\right)}^2} \cdot U_{in} \cdot \Delta R_{11} \\
                    &= \frac{10^6~\Omega}{{\left(10^4~\Omega+10^6~\Omega\right)}^2} \cdot 10~V \cdot 10^4 \cdot 0,05 + \frac{10^4~\Omega}{{\left(10^4~\Omega+10^6~\Omega\right)}^2} \cdot 10~V \cdot 10^6 \cdot 0,05 \\
                    &= 10~mV
\end{align*}

Die genauen Toleranzwerte der Widerstände sind leider nicht bekannt, deshalb wird von großzügig gewählten 5\% ausgegangen, obwohl typischerweise in solchen Schaltungen 1\% Widerstände verbaut werden.

Die Rechnung ergibt \(U_{ref} = \left(99 \pm 10\right)~mV\).

Dies liegt somit bei größtmöglichem Fehler immernoch 9~mV über der gemessenen Untergrenze des Komparators und kann somit nicht der einzige Grund für die Abweichung sein.

Da leider nicht genau aus der Versuchsbeschreibung herausgeht, welche Art des Operationsverstärkers TL081 verbaut ist, werden im folgenden, jeweils die maximalen Werte verwendet.

Rechnet man nun allerdings auch noch zu diesem Fehler von 10~mV der Referenzspannung die maximale Input-Offset Spannung des TL081 \(V_{IO,max} = 15~mV\)~\cite{Diode} bei Raumtemperatur hinzu,
 so ergibt dies einen Fehler von \(\pm 25~mV\), wodurch die gemessenen \(80~mV\) im Toleranzbereich \(99~mV~\pm~25~mV\) des OPVs liegen. 

Die Input-Offset Spannung ist hierbei der Spannungsunterschied, welche die beiden Eingänge des OPVs aufweisen müssen, um von diesem als über bzw.\ unter der Referenzspannung erkannt zu werden.
Dies ergibt somit einen plausiblen Fehler des Operationsverstärkers, welcher aufgrund von minimalen Unterschieden in der Fertigung zwischen den beiden Kontaktpfaden entsteht.~\cite{OpAmp} 

	\section{Literaturverzeichnis}

	\subsection{Quellenverzeichnis}
	\printbibliography[heading=none]

	\subsection{Abbildungsverzeichnis}
	\listoffigures

	\subsection{Tabellenverzeichnis}
	\listoftables

	
\end{document}
