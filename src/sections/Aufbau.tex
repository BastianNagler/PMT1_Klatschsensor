\section{Versuchsaufbau}\label{section:aufbau}

\subsection{Schematischer Aufbau}\label{subsection:schematischer_aufbau}

\begin{figure}[H]
    \includegraphics[width=\textwidth]{../Bilder/Aufbau.png}
    \caption{Schematischer Aufbau der Schaltung~\cite{Beschreibung}}\label{fig:aufbau}
\end{figure}

Der in der Abbildung~\ref{fig:aufbau} rot dargestellte Bereich zeigt das Mikrofon inklusive einer Gleichspannungsquelle, welche eine Arbeitspunktverschiebung des Mikrofons bewirkt.
Diese Gleichspannung dringt allerdings nicht in den weiteren Signalverarbeitungsweg ein, da sie durch den Kondensator C1 gefiltert wird.

Da die Signale des Mikrofons sehr schwach sind, müssen diese zunächst verstärkt werden.
Hierfür wird ein invertierender Operationsverstärker (gelber Bereich) verwendet.
Die Verstärkung des OPVs wird durch die Widerstände \( R_2 \) und \( R_3 \) bestimmt und beträgt in der Grundkonfiguration:~\cite{Beschreibung}
\begin{align*}
    V = - \frac{R_3}{R_2} = - \frac{56~k\Omega}{2,2~k\Omega} \approx -25,45
\end{align*}

Das verstärkte Signal wird anschließend durch den Einweggleichrichter, implementiert durch die Diode \(D_1\), gleichgerichtet.

Die in grün gekennzeichneten Bauteile (\(C_2\) und \(R_5\)) bewirken eine Glättung des gleichgerichteten Signals,
 um die Spannungslücken des Signals zu überbrücken und außerdem auch um die Haltezeit der LED am Ende zu erhöhen.
Dies geschieht, indem der Kondensator durch das Signal aufgeladen wird und sich daraufhin langsam über \(R_5\) entlädt.

Diese Ausgangsspannung wird dann über \(R_6\) an den positiven Eingang eines Komparators, welcher durch einen OPV realisiert wird, weitergeleitet.
Der Komparator vergleicht die Eingangsspannung mit einer Referenzspannung, die über das Spannungsteilernetzwerk aus \(R_7\) und \(R_{11}\) festgelegt wird.
Ist die Eingangsspannung \(U_{EK}\) größer als die Referenzspannung \(U_{Ref}\), schaltet der Komparator seinen Ausgang auf eine Spannung nahe der Versorgungsspannung.
Ist \(U_{EK} < U_{Ref}\), so beträgt die Ausgangsspannung einen Wert nahe der negativen Versorgungsspannung.

Der Ausgang des Komparators schaltet über den Vorwiderstand \(R_9\) die LED \(D_2\),
 welche aufgrund der Eigenschaft der Diode nur bei einer positiven Spannung des Komparator-Ausgangs aufleuchtet.

\subsection{Realer Aufbau}\label{subsection:realer_aufbau}

\begin{figure}[H]
    \includegraphics[width=\textwidth]{../Bilder/realer_aufbau.jpg}
    \caption{Realer Aufbau der Schaltung}\label{fig:realer_aufbau}
\end{figure}

Der reale Aufbau wirkt unspektakulär. Die gesamte Schaltung ist nicht einsehbar unter einer Platte realisiert, lediglich die Widerstände \(R_3\) und \(R_5\) liegen oberhalb, um diese verändern zu können.
Außerdem sind Anschlüsse für die Versorgungsspannung (-10~V, 0~V und +10~V) sowie Anschlüsse für das Oszilloskop an den gewünschten Messpunkten vorhanden.

In Abbildung~\ref{fig:realer_aufbau} sind im wesentlichen vier verwendete Geräte zu erkennen:

\begin{enumerate}
    \item Netzteil mit zwei seperaten Spannungsquellen
    \item Eigentliche Schaltung des Klatschsensors
    \item Funktionsgenerator Rigol DG1012
    \item Oszilloskop Rigol DS1102E
\end{enumerate}

Die Spannungsquellen werden innerhalb des Netzteils in Serie geschaltet, um die benötigte bipolare Spannung von \(\pm 10~V\) zu erhalten.
Dabei wird der Minuspol auf -10~V, der gemeinsame Pol auf 0~V und der Pluspol der Quellen auf +10~V gelegt.