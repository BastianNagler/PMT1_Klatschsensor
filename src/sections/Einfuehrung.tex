\section{Versuchseinführung}

Im Versuch Klatschsensor wird eine bereitgestellte Schaltung analysiert, die ein akustisches Signal --- etwa ein Klatschen --- erfasst,
 verstärkt und damit eine LED schaltet.
Dabei werden grundlegende messtechnische und elektronische Konzepte praktisch angewendet.

Zentrale Bestandteile sind ein Elektretmikrofon, ein Operationsverstärker zur Signalverstärkung,
 eine Diode zur Einweggleichrichtung sowie ein Komparator, der die Spannung mit einer Referenz vergleicht und bei Überschreitung eine LED ansteuert.

Durch die Arbeit mit dem Digitaloszilloskop und dem Funktionsgenerator wird der Umgang mit typischen Messgeräten vertieft.
Zudem wird die Funktion und Anwendung des invertierenden Verstärkers und des Komparators untersucht.
Ein weiterer Schwerpunkt liegt in der Analyse von Lade- und Entladevorgängen eines Kondensators, die das zeitliche Verhalten des Sensors bestimmen.
Des weiteren wird die Klassifizierung von Widerständen anhand ihres Farbcode-Systems über den gesamten Versuch hinweg behandelt.\ \cite{Beschreibung}

Außerdem wurde während der Versuchsdurchführung großer Wert auf die kritische Betrachtung der Messergebnisse gelegt und es wurden unter den Gruppen Fehlerquellen identifiziert und diskutiert.